\documentclass{article}
\usepackage[margin=.5in]{geometry}
\usepackage{graphicx, dblfloatfix}
\usepackage{amsmath, amssymb, amsfonts, mathrsfs, mathtools}
\usepackage[english]{babel}
\usepackage[autostyle, english = american]{csquotes}
\usepackage[normalem]{ulem}
\usepackage[title,titletoc,toc]{appendix}
\usepackage{array, booktabs, colortbl}
\MakeOuterQuote{"}


\newcommand{\redchi}{$\tilde{\chi}^2\,$}
\DeclareMathOperator{\erf}{erf}
\DeclareMathOperator{\cov}{cov}
\DeclarePairedDelimiter\abs{\lvert}{\rvert}%
\DeclarePairedDelimiter{\parens}{\lparen}{\rparen}

\title{Optical Pumping}
\author{Aman LaChapelle}

\begin{document}
\raggedright
\maketitle

\begin{abstract}
  We will show that doppler broadening due to thermal fluctuations can be alleviated in order to show spectral features that would otherwise be hidden.  In this case, we will show measurement of the hyperfine states of Rubidium (whose linewidth requires higher resolution than the doppler broadening allows) in order to demonstrate this technique.
\end{abstract}

\tableofcontents
\newpage

\section{Introduction}
  We show that 

\end{document}
