\documentclass{article}
\usepackage[margin=.5in]{geometry}
\usepackage{graphicx, dblfloatfix}
\usepackage{amsmath, amssymb, amsfonts, mathrsfs, mathtools}
\usepackage[english]{babel}
\usepackage[autostyle, english = american]{csquotes}
\usepackage[normalem]{ulem}
\usepackage[title,titletoc,toc]{appendix}
\usepackage{array, booktabs, colortbl}
\MakeOuterQuote{"}


\newcommand{\redchi}{$\tilde{\chi}^2\,$}
\DeclareMathOperator{\erf}{erf}
\DeclareMathOperator{\cov}{cov}
\DeclarePairedDelimiter\abs{\lvert}{\rvert}%
\DeclarePairedDelimiter{\parens}{\lparen}{\rparen}

\title{Optical Pumping}
\author{Aman LaChapelle}

\begin{document}
\raggedright
\maketitle

\begin{abstract}
  We will show that doppler broadening due to thermal fluctuations can be alleviated in order to show spectral features that would otherwise be hidden.  In this case, we will show measurement of the hyperfine states of Rubidium (whose linewidth requires higher resolution than the doppler broadening allows) in order to demonstrate this technique.
\end{abstract}

\tableofcontents
\newpage

\section{Introduction}
  We show that it is possible to spectally resolve features of the Hyperfine structure of Rubidium that would otherwise be impossible outside some sort of atomic trap using a technique called doppler-free spectroscopy.  We also show that it is possible to make use of the interference pattern of a Michelson Interferometer to make calibrations and create a conversion from time into frequency for the purposes of measurement on a time-resolved oscilloscope.  This means that we can use an oscilloscope with (potentially) much better-resolved time dynamics to mimic the purpose of a spectrum analyzer.

  \hspace{.25cm}

  In particular, we discuss the methods used for finding the information required, including the methods used to find the Hyperfine transitions, measure the linewidth of the resonances, and the linewidth of the doppler-broadened peak.

\section{Theory} %%%%%%%%%%%%%%%%%%%%%%%%

\end{document}
