\documentclass{article}
\usepackage[margin=.5in]{geometry}
\usepackage{graphicx, dblfloatfix}
\usepackage{amsmath, amssymb, amsfonts, mathrsfs, mathtools}
\usepackage[english]{babel}
\usepackage[autostyle, english = american]{csquotes}
\usepackage[normalem]{ulem}
\usepackage[title,titletoc,toc]{appendix}
\usepackage{pgfplotstable}
\usepackage{array, booktabs, colortbl}
\MakeOuterQuote{"}

\pgfplotsset{compat=1.12}


\newcommand{\redchi}{$\tilde{\chi}^2\,$}
\DeclareMathOperator{\erf}{erf}
\DeclareMathOperator{\cov}{cov}
\DeclarePairedDelimiter\abs{\lvert}{\rvert}%
\DeclarePairedDelimiter{\parens}{\lparen}{\rparen}

\title{Optical Pumping}
\author{Aman LaChapelle}

\begin{document}
\raggedright
\maketitle

\begin{abstract}
  We investigate the effects of an applied external field on Rubidium-87 atoms, known as Zeeman splitting.  We will further investigate the process known as optical pumping, which is used to excite electrons from the $^2S_{1/2},\, f = 2$ state to the $^2P_{1/2},\, f = 2$ state with the goal of creating a fully pumped state, characterized by a majority of electrons in the $^2S_{1/2},\, f = 2,\, m_f = 2$ state.  Using this pumped state and the requrements surrounding its creation, we will further investigate stimulated emission of photons as well as Larmor Precession, which occurs under another special set of circumstances.
\end{abstract}

\tableofcontents
\newpage

\section{Introduction}%%%%%%%%%%%%%%%%%%%%%%%%%%%%%%%%%%%%%%%%%%%%%%%%%%%%%%%%%
  putting things here I don't know what to put here
\section{Theory}%%%%%%%%%%%%%%%%%%%%%%%%%%%%%%%%%%%%%%%%%%%%%%%%%%%%%%%%%%%%%%%
  fill this with zeeman effect, stimulated emission, boltzmann distribution/bose-einstein statistics, RF de-pump and Larmor precession
\section{Apparatus/Procedure}%%%%%%%%%%%%%%%%%%%%%%%%%%%%%%%%%%%%%%%%%%%%%%%%%%
  answer question(s) about apparatus, include picture
\section{Data/Uncertainty Analysis}%%%%%%%%%%%%%%%%%%%%%%%%%%%%%%%%%%%%%%%%%%%%

\section{Conclusion}%%%%%%%%%%%%%%%%%%%%%%%%%%%%%%%%%%%%%%%%%%%%%%%%%%%%%%%%%%%

% \begin{thebibliography}
%
%
% \end{thebibliography}

\end{document}
